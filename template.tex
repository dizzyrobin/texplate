\documentclass[a4paper,spanish,12pt]{report}
  % pre\'ambulo

  \usepackage{lmodern}
  \usepackage[T1]{fontenc}
  \usepackage[spanish,activeacute]{babel}
  \usepackage[utf8]{inputenc}
  \usepackage{mathtools}
  \usepackage[left=2.5cm,top=2.5cm,right=2.5cm,bottom=2.5cm]{geometry}
  \usepackage{relsize}
  \usepackage{float}
  \usepackage{listings}
  \usepackage{color}
  \usepackage{enumitem}
  \usepackage{blindtext}
  \usepackage{amsmath}
  \usepackage{fixltx2e}
  \usepackage[]{hyperref}

  % Define vars
  \newcommand{\textitle}{Título de la memoria}
  \newcommand{\texsubject}{Asignatura de ejemplo}
  \newcommand{\texyear}{2018/19}
  \newcommand{\texmonth}{Enero}
  \newcommand{\texdate}{1 de Enero del 2019}
  \newcommand{\texgroup}{Grupo 3.x}

  \hypersetup{
    pdftitle={\textitle}
    pdfauthor={Marta Ortega Soto y Pedro Javier Nicol\'as Zamora},
    pdfsubject={documentation},
    bookmarksnumbered=true,
    bookmarksopen=true,
    bookmarksopenlevel=1,
    colorlinks=true,
    pdfstartview=Fit,
    pdfpagemode=UseOutlines,
    pdfpagelayout=TwoPageRight
  }

  % Code break page
  \newcommand{\cbp}[3]{\lstinputlisting[firstline=#2, lastline=#3, firstnumber=#2]{#1}}
  %\cbp{FILE}{FIRSTLINE}{LASTLINE}


  \selectlanguage{spanish}

  \newcommand{\tab}[1]{\hspace{.03\textwidth}\rlap{#1}}

  \definecolor{gray}{rgb}{0.5,0.5,0.5}

  \lstset{
    frame=single,
    extendedchars=true,
    tabsize=4,
    breaklines=true,
    postbreak=\raisebox{0ex}[0ex][0ex]{\ensuremath{\color{red}\hookrightarrow\space}},
    % SQL
    % morekeywords={SELECT, FROM, WHERE, LEFT, RIGHT, JOIN, UNION, MINUS, HAVING, ORDER, BY, CASE, NOT, IN, AND, OR, NULL, GROUP, AS, ON, IS, Clave, ajena, primaria, Admiten, nulos, Derivados, CREATE, TABLE, ASSERTION, CONSTRAINT, CHECK, DELETE, CASCADE, PRIMARY, KEY, UNIQUE, INSERT, INTO, VALUES, COMMENT, COLUMN, VIEW, SET, ALTER, ROLLBACK, DROP, UPDATE, REFERENCES, FOREIGN},
    % C++
    % morekeywords={return, include, if, for, while}
    morekeywords={return, include, if, for, while}
    numbers=left,
    basicstyle=\footnotesize,
    numberstyle=\tiny\color{gray}
  }

  \makeatletter
  \def\@makechapterhead#1{%
    \vspace*{50\p@}%
    {\parindent \z@ \raggedright \normalfont
      \interlinepenalty\@M
      \Huge\bfseries  \thechapter.\quad #1\par\nobreak
      \vskip 40\p@
    }}
  \makeatother

  \title{\textitle}
  \author{Pedro Javier Nicol\'as Zamora}

  \begin{document}
  % cuerpo del documento

  \begin{titlepage}
  \begin{center}
  \begin{figure}[htb]
  \end{figure}
  \begin{Huge}
  \textbf{Universidad de Murcia}\\
  \end{Huge}
  \begin{huge}
  \vspace{0.1cm}
  \texsubject \\
  \end{huge}
  \begin{LARGE}
  Curso \texyear\ - Convocatoria de \texmonth \\
  \end{LARGE}
  \vspace*{2cm}
  \begin{Huge}
  \textbf{\textitle} \\
  \end{Huge}
  \vspace*{10cm}
  \rule{80mm}{0.1mm}\\
  \vspace*{0.5cm}
  \begin{Large}
  % \ \ \ \ Marta Ortega Soto\ \ \ \ \ \ Pedro Javier Nicolás Zamora\\
  % 23314500\textit{K}\ \ \ \ \ \ \ \ \ \ \ \ \ \ \ \ \ \ \ \ \ 49224983\textit{S}\ \ \ \ \\
  % marta.ortega3@um.es\ \ \ \ \ \ pedrojavier.nicolas@um.es\\

  % Me
  Pedro Javier Nicolás Zamora\\
  49224983\textit{S}\ \ \ \ \\
  pedrojavier.nicolas@um.es\\

  \vspace*{0.5cm}
  \texgroup \\
  \texdate \\
  \end{Large}
  \end{center}
  \end{titlepage}

  \setcounter{page}{2}

  \tableofcontents
  \pagebreak

  %\listoffigures
  %\pagebreak

  % Tags:
  %\chapter{Chapter}
  %\section{Section}
  %\subsection{Subsection}

  % Insert image:
  %\begin{figure}[H]
    %\begin{center}
      %\includegraphics[width=0.5\textwidth]{img/av-tabla-contraste.png}
    %\end{center}
    %\caption{Algoritmo Voraz. Tabla de tiempos de ejecución contraste de estudio teórico y experimental.}
  %\end{figure}

  % Insert code #1:
  %\begin{figure}[H]
  %\lstinputlisting{psc/705/AlgoritmoVoraz.psc}
  %\caption{Pseudocódigo Algoritmo Voraz.}
  %\end{figure}

  % Insert code #2:
  % \begin{figure}[H]
  % \cbp{src/P2.sql}{1}{35}
  % \caption{Creación de la base de datos (Parte 1)}
  % \end{figure}

  \chapter{Chapter}
  \section{Section}
  \subsection{Subsection}

  Hello world!



  \end{document}
